
% You may delete everything from \appendix up to \end{document} if you don't need it.
\appendix

\chapter{Appendix}

\section{Small vs Large Vocabulary}\label{app:small_vs_large_vocabulary}

\section{Chord Mapping}\label{app:chord_mapping}

Chords in Harte notation were mapped to the vocabulary with $C=170$ by first converting them to a tuple of integers using the Harte library. These integers represent pitch classes and are in the range 0 to 11 inclusive. They are transposed such that 0 is the root pitch. These pitch classes were then matched to the pitch classes of a quality in the vocabulary, similar to the work by \citet{StructuredTraining}. However, for some chords, this was not sufficient. For example, a \texttt{C:maj6(9)} chord would not fit perfectly with any of these templates due to the added 9th. Therefore, the chord was also passed through Music21's~\citep{music21} chord quality function which matches chords such as the one above to major. This function would not work alone as its list of qualities is not as rich as the one defined above. If the chord was still not matched, it was mapped to \texttt{X}. This additional step is not done by \citet{StructuredTraining} but gives more meaningful labels to roughly one third of the chords previously mapped to \texttt{X}.

\section{Accuracy vs Context Length of Evaluation}\label{app:accuracy_vs_context_length}

\begin{figure}[H]
    \centering
    \includegraphics[width=0.5\textwidth]{figures/context_length_vs_accuracy.png}
    \caption{Accuracy vs context length of evaluation. The accuracy increases very slightly. The effect size is so small that we conclude it does not make a difference, and choose to evaluate over the entire song at once.}
    \label{fig:accuracy_vs_context_length}
\end{figure}

\section{Accuracy vs Hop Length}\label{app:accuracy_vs_hop_length}

\begin{figure}[H]
    \centering
    \includegraphics[width=0.5\textwidth]{figures/hop_length_vs_accuracy.png}
    \caption{Accuracy vs hop length. Metrics are not directly comparable over hop lengths due to different likelihoods. However, the metrics are fairly consistent over different hop lengths, certainly over the region explored by the literature $[512,2048,4096]$. Every hop length tested is short enough to be more granular than chords, but not so short that the computed CQT is too noisy. We continue with the default hop length of $4096$, to be consistent with some of the literature while keeping computational cost low.}
    \label{fig:accuracy_vs_hop_length}
\end{figure}

\section{Incorrect Region Lengths With/Without Smoothing}\label{app:histogram_over_region_lengths}

\begin{figure}[H]
    \centering
    \includegraphics[width=0.6\textwidth]{figures/incorrect_region_smoothing_histogram.png}
    \caption{Histogram over incorrect region lengths for a \emph{CRNN} with and without smoothing. An incorrect region is defined as a sequence if incorrect frames with correct adjacent of either end. Both distributions have a long-tail, with $26.7\%$ regions being of length 1 without smoothing. This raises concerns over the smoothness of outputs and requires some form of post-processing explored in Section~\ref{sec:decoding}. The distribution is more uniform with smoothing, with approximately half the very short incorrect regions.}
    \label{fig:histogram_over_region_lengths}
\end{figure}
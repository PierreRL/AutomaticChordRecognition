\begin{preliminary}

    \title{Chord Recognition with Deep Learning}
    
    \author{Pierre Lardet}
    
    \course{Computer Science and Mathematics}
    
    \project{4th Year Project Report}
    
    
    \date{\today}
    
    \abstract{
      Progress in automatic chord recognition has been slow since the advent of deep learning in the field. In order to understand why, I conduct experiments on existing methods and test hypotheses enabled by recent developments in generative models. Findings show that chord classifiers still have poor performance on rare chords and that pitch augmentation provides the biggest boost in accuracy. Features extracted from generative models do not help but synthetic data presents an exciting avenue for future work. I conclude by improving the interpretability of model outputs with beat detection, reporting some of the best results in the field and providing qualitative analysis. Much work remains to be done to solve automatic chord recognition but I hope that this thesis charts a path for others to try.
    }
    
    \maketitle
    
    \newenvironment{ethics}
       {\begin{frontenv}{Research Ethics Approval}{\LARGE}}
       {\end{frontenv}\newpage}
    
    \begin{ethics}
    This project was planned in accordance with the Informatics Research
    Ethics policy. It did not involve any aspects that required approval
    from the Informatics Research Ethics committee.
    
    \standarddeclaration
    \end{ethics}
    
    
    \begin{acknowledgements}
    Thank you to my two lovely supervisors, \hyperlink{https://homepages.inf.ed.ac.uk/amos/index.html}{Prof.~Amos Storkey} and \hyperlink{https://www.acoustics.ed.ac.uk/people/dr-alec-wright/}{Dr.~Alec Wright} for allowing me to fulfil my desire for such a personally motivating project. They offered guidance on this project and research at large which I will take forward. I would like to thank Andrea Poltronieri for sharing the dataset used in this work. Finally, I would like to thank my friends and family for providing me with reasons to take a break.
    \end{acknowledgements}
    
    
    \tableofcontents
    \end{preliminary}
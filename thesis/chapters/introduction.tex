\chapter{Introduction}

Chords form an integral part of music. Part of how musicians understand music is through harmonic structure. Chord annotations allow music to be easily shared, performed, improvised and analysed. Not all chord annotations available online are free or of a high enough quality. This is because creating high quality chord annotations requires a trained musician. 

To this end, I investigate the use of deep learning for automatic chord recognition. Data-driven methods have dominated the field of all automatic music transcription for over a decade. However, the significant progress of early models has not continued in recent years while the problem remains far from solved.

In this work, I first aim to understand why. I implement a common benchmark model and conduct a thorough analysis of its behaviour. This involves looking at common mistakes the model makes, performance on rarer chords and how its predictions relate to time. I then use these observations to study the various ways models have been improved and compare them. 

I also conduct novel research on the use of generative models as both feature extractors and a source of new data. This is enabled by chord-conditioned generative models developed in recent years. I conclude by re-thinking how the model predicts chords in time by incorporating beat estimation. 

The analysis of existing models, exploration of improvements and discussion of new research directions constitutes a novel contribution to the field of automatic chord recognition. Despite lack of performance improvements in recent years, I hope that this work provides motivation for others to continue pursuing research aiming to solve the problem posed by chord recognition. All code is available on Github.\footnote{\url{https://github.com/PierreRL/LeadSheetTranscription}} Data can be made available upon request.\footnote{lardet[dot]pierre[at]gmail.com}

\section{Outline}

The thesis is structured as follows:

\begin{itemize}
    \item \textbf{Chapter 2} provides background information on harmony and chord recognition, pointing out trends in the field and the most exciting avenues for research.
    \item \textbf{Chapter 3} describes the datasets, evaluation metrics and training procedure used in this work.
    \item \textbf{Chapter 4} contains the implementation of a model from the literature and an analysis of its properties and predictions. 
    \item \textbf{Chapter 5} extends this work by studying various methods of improving on this model. Some of these improvements constitute deep analysis of existing improvement while others represent novel avenues of research.
    \item \textbf{Chapter 6} concludes the thesis and provides suggestions for future work.
\end{itemize}
\chapter{Introduction}

\section{Motivation}

TODO:
Rewrite introduction to focus on chord recognition.

- Useful for musicians
- Useful for musicologists
- Connection to lead sheets

Chords form an integral part of music. Used in music of all forms. Used for research

To this end, we investigate the use of deep learning for automatic chord recognition. Data-drive methods have dominated the field of automatic music transcription in recent years, and have shown great promise for chord recognition. However, progress has not been made since 2015. Why? What might be done to improve?

We conduct a thorough analysis of the state-of-the-art models for automatic chord recognition, and investigate methods of improving on these models. We look at the ways others have improved these models and compare and contrast them. 

We also take inspiration from other fields of music transcription and leverage modern generative models to provide new representations training data and generate synthetic data itself.


\section{Aims}

The aims of this project are:
\begin{itemize}
    \item Compare state-of-the-art models for automatic chord recognition.
    \item Conduct a thorough analysis of the models and their performance.
    \item Investigate methods of improving on these models.
    \item To perform experiments with new sources of data, data augmentation and synthetic data generation.
\end{itemize}

\section{Outline}

The report is structured as follows:

\begin{itemize}
    \item \textbf{Chapter 2} provides background information on chord transcription and related work.
    \item \textbf{Chapter 3} describes the datasets, evaluation metrics and training procedure used in this project.
    \item \textbf{Chapter 4} compares various models from the literature and investigates improvements.
    \item \textbf{Chapter 5} extends this work with synthetic data generation and compares results on a new dataset.
    \item \textbf{Chapter 6} concludes the report and provides suggestions for future work.
\end{itemize}
\chapter{Introduction}

Chords form an integral part of music. Part of how musicians understand music is through harmonic structure. Chord annotations are a symbolic representation of the chords in a piece of music. They allow music to be easily shared, performed, improvised and analysed. Not all chord annotations available online are free or of a high enough quality because creating high-quality chord annotations requires a trained musician. 

To this end, I investigate the use of deep learning in automatic chord recognition to create chord annotations of music. Data-driven methods have dominated the field for over a decade. However, the significant progress of early models has not continued in recent years; the problem remains far from solved.

In this work, I first aim to understand why performance improvements have stagnated. I implement a standard benchmark model and conduct a thorough analysis of its behaviour. This involves looking at the model's common mistakes, performance on rarer chords and how its predictions relate to time. I then use these observations to study different methods of improving these models. 

I also conduct novel research on using generative models as both feature extractors and a source of new data. This is enabled by chord-conditioned generative models developed in recent years. I conclude by rethinking how the model predicts chords in time by incorporating beat estimation. 

This work goes towards enabling software which can be used to better understand, create and learn music. Easily accessible and accurate chord recognition models would allow producers to better understand their work and musicologists to study larger datasets. Musicians and hobbyists could access chord annotations for their favourite songs or analyse their performances and improvisations.

The analysis of existing models, exploration of improvements and discussion of new research directions constitute a novel contribution to the field of automatic chord recognition. Despite the lack of performance improvements in recent years, I hope this work motivates others to continue pursuing research aiming to solve the problem posed by automatic chord recognition. 

\section{Outline}

The thesis is structured as follows:

\begin{itemize}
    \item \textbf{Chapter 2} provides background information on harmony, chord recognition and musical data. I then discuss existing literature on the subject, pointing out trends in the field and the most exciting avenues for research.
    \item \textbf{Chapter 3} describes the datasets, evaluation metrics and training procedure used.
    \item \textbf{Chapter 4} contains the implementation of a convolutional neural network from the literature, followed by an analysis of its properties and predictions. I observe behaviours which provide opportunities to improve the model. 
    \item \textbf{Chapter 5} extends this work by studying various methods of improvement. Some of these experiments analyse existing improvements, while others present novel avenues of research.
    \item \textbf{Chapter 6} concludes the thesis and provides suggestions for future work.
\end{itemize}

\vspace{0.5cm}

All code is available on GitHub.\footnote{\url{https://github.com/PierreRL/AutomaticChordRecognition}} Data can be made available upon request.\footnote{lardet[dot]pierre[at]gmail.com}
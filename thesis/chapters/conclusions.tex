
\chapter{Conclusions and Further Work}

\section{Conclusions}

- What do the results say? What did we find?

\section{Limitations}
Limitations:
- Genre
- Standard tuning, Western
- No lyrics
- Size of vocabulary: inversion etc
- Some labels don't have a clear meaning

\section{Further Work}
Further work:
- More detailed expts:
    - more models for gen features, more varied, with vocals etc
    - better future chord conditioned models for synthetic data
- More data e.g. HookTheory
- Better beat tracking?
- Jointly predicting chord segmentation, cite 20 years and cite Choco people for inverse problem.

- Incorporate functional harmony or chord vectors as targets
- Better understanding of the glass ceiling, human inter-annotator scores.

Finally, chord annotations are inherently subjective. Inter-annotator agreement of the root of a chord is estimated at lying between 76\%~\citep{AnnotatorAgreement76} and 94\%~\citep{RockHarmonyAnalysis94} but these metrics are calculated using only four and two annotators respectively. These agreement estimates use the same metrics defined in Section~\ref{sec:evaluation}. Furthermore, \citet{FourTimelyInsights} and \citet{UnderstandingSubjectivity} posit that agreement between annotations can be far lower for some songs. Little has been done to address
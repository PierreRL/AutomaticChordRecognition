\chapter{Introduction}

\section{Motivation}

% Lead sheets are a common way of representing music in a simple and concise format. They consist of an aligned melody and harmony through chord symbols. Lead sheets are widely used in the music industry as they provide a quick and easy way to learn and perform music, especially in jazz music where improvisation is common. Lead sheets are also used in music education, as they provide a simple way to introduce students to music theory and notation. However, creating lead sheets can be a time-consuming and error-prone process for complex pieces of music. This project aims to develop a model that can automatically generate lead sheets from audio recordings of music, which could be used to help musicians, educators, and students create lead sheets more easily and accurately. 

% However, there is no good source of lead sheets for many songs. Popular websites such as Ultimate Guitar [XX] and HookTheory [XX] provide user-submitted chord annotations for almost any song, but the quality of these annotations can vary significantly [XX]. Tools like Chordify [XX] can automatically generate chord annotations from audio recordings, but it is paid and designed for popo music. Online forums where a variety of leads can be found do exist, such as MusicNotes [XX] and SheetMusicDirect [XX], but are expensive, have limited selections and are also user submitted, meaning their quality cannot be guaranteed.

% To this end, we investigate the use of machine learning models for lead sheet generation, requiring the model to perform automatic chord recognition and melody transcription.

TODO:
Rewrite introduction to focus on chord recognition.

- Useful for musicians
- Useful for musicologists
- Connection to lead sheets

\section{Aims}

The aims of this project are:
\begin{itemize}
    \item Compare state-of-the-art models for automatic chord recognition.
    \item Investigate methods of improving on these models 
    \item To investigate the use of synthetic data generation for improving the performance of the model.
\end{itemize}

\section{Outline}

The report is structured as follows:

\begin{itemize}
    \item \textbf{Chapter 2} provides background information on chord transcription and related work.
    \item \textbf{Chapter 3} describes the datasets, evaluation metrics and training procedure used in this project.
    \item \textbf{Chapter 4} compares various models from the literature and investigates improvements.
    \item \textbf{Chapter 5} extends this work with synthetic data generation and compares results on a new dataset.
    \item \textbf{Chapter 6} concludes the report and provides suggestions for future work.
\end{itemize}
\chapter{Introduction}

\section{Motivation}

Lead sheets are a common way of representing music in a simple and concise format. They consist of an aligned melody and harmony through chord symbols. Lead sheets are widely used in the music industry, as they provide a quick and easy way to learn and perform music, espeically in jazz music where improvisation is common. Lead sheets are also used in music education, as they provide a simple way to introduce students to music theory and notation. However, creating lead sheets can be a time-consuming and error-prone process, especially for complex pieces of music. This project aims to develop a model that can automatically generate lead sheets from audio recordings of music, which could be used to help musicians, educators, and students create lead sheets more easily and accurately. 

However, there is no good source of lead sheets for many songs. Popular websites such as Ultimate Guitar [XX] and HookTheory [XX] provide user-submitted chord annotations for almost any song, but the quality of these annotations can vary significantly [X]. Tools like Chordify [XX] can automatically generate chord annotations from audio recordings, but lack aligned melodies and have formats designed for complete beginners. Online forums where a variety of leads can be found do exist, such as MusicNotes [XX] and SheetMusicDirect [XX], but are expensive, have limited selections and are also user submitted, meaning their quality cannot be guaranteed.

To this end, we investigate the use of machine learning models for lead sheet generation, requiring the model to perform automatic chord recognition and melody transcription, and align them in order to generate a lead sheet.

\section{Aims}

The aims of this project are:
\begin{itemize}
    \item To develop a model that can generate lead sheets from audio recordings of music.
    \item To implement and evaluate current state-of-the-art models for automatic chord recognition and melody transcription.
    \item Investigate the use of different data representations for music transcription and generation.
    \item To evaluate the performance of the model on a variety of music genres and styles.
    \item To investigate the use of different models/datasets/data augmentation methods for automatic chord recognition, melody transcription and lyric transcription.
    \item To build a useable, reproducible tool that can generate lead sheets from audio recordings of music.
\end{itemize}

\section{Outline}

The report is structured as follows:

\begin{itemize}
    \item \textbf{Chapter 2} provides background information on lead sheets, music transcription, and related work.
    \item \textbf{Chapter 3} describes the experiments conducted in this project, including the data, models, training, and evaluation.
    \item \textbf{Chapter 4} presents and analyses the results of the experiments.
    \item \textbf{Chapter 5} concludes the report and provides suggestions for future work.
\end{itemize}